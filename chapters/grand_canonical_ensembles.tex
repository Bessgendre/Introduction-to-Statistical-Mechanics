
\chapter{巨正则系综}\label{cha:巨正则系综}

在上一章中,我们基于微正则系综 $(N,V,E)$ 的三个广度决策变量的测量没有实际可行性的考虑,首先对能量 $E$ 通过勒让德变换将其等价转换为了熵对能量的导数:也就是强度量 $T$ ,根据我们的实际经验,将它命名为“温度”。基于此,我们将使用 $(N,V,T)$ 作为决策变量的这一方法称为正则系综。这种理论处理方案的有效性,从已经讨论过的许多例子就可以看得很清楚。然而,对于一些物理和化学问题,正则系综表述形式的适用性毕竟还是有所限制的——不仅系统的能量难以直接测量得到,粒子数目 $N$ 亦是如此。

现在,我们在对能量 $E$ 勒让德变换后,继续对另一个广度性质 $N$ 做勒让德变换。再次回想起以熵 $S$ 为核心的热力学公式
\[
    dS = \frac{1}{T} dE + \frac{P}{T} dV - \frac{\mu}{T} dN
\]

所以,和在正则系综中的思路一样,与 $N$ 等价的强度量为 $\mu$ ,这也就是本章的主题: \textcolor{RoyalBlue}{\textbf{\kaishu 巨正则系综}}。

由于确定热力学系统的一个平衡态至少需要一个广度性质,所以最多只能使用两次勒让德变换。如果不选择 $N$ 而是对 $V$ 做变换,就会得到 \textcolor{RoyalBlue}{\textbf{\kaishu 等温等压系综}}。

\section{巨正则系综的研究方法}\label{sec:巨正则系综的研究方法}

\subsection{统计量的物理含义}\label{sec:统计量的物理含义}

\subsection{粒子数与能量的涨落}\label{sec:粒子数与能量的涨落}

\section{实例}\label{sec:巨正则实例}

\section{总结}\label{sec:巨正则总结}

\begin{justification}{\kaishu 反思与质疑}
\kaishu \fontsize{11pt}{16pt}
    \quad\quad 似乎我们的理论在巨正则系综这里走到了尽头,但是不要忘记:最开始的三个广度性质并不一定要选择 $(N,V,E)$ ,因为广度性质的背后,是系统的独立可加运动积分,而这样的可加运动积分一共有七个:能量、动量的三个分量以及角动量的三个分量(参见(\ref*{sec:分布函数})节)。所以在原则上,我们完全可以选择可加运动积分的线性组合作为描述系统的状态参量,也就是最优化问题的决策变量。

    \quad\quad 力学系统运动积分是由哈密顿函数确定的,但哈密顿力学本身并不局限于物理。它对称共轭的形式使得人们可以将一些本来和力学相距甚远的问题(比如捕食者-猎物系统中种群数量随时间的变化)通过合适的变换统一在哈密顿体系下。
    
    \quad\quad 而系综理论,作为哈密顿力学的统计延伸,同样不受物理的束缚。也就是说,如果今天有了一个现实系统,而经验告诉我们:为了确定系统的状态需要四个状态参量,其中要有一个广度性质。那么这时,巨正则系综是不是还可以更进一步,再做一组勒让德变换呢?
\end{justification}