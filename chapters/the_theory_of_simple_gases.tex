\chapter{简单气体理论}

前面三章我们分别对三种系综分别进行了一定的讨论。它们在“包含系统全部的运动信息”的意义上都是等价的,区别仅在将不易测量的广度决策变量通过勒让德变换转化为了容易测量的强度决策变量,而这一变换过程对于热力学系统而言,停留在了巨正则系综上。这标志着我们研究的基本原理与基本框架已经成型,现在我们来将它运用于一个最简单的系统,正如它的名称所标明的那样:简单气体。

首先,研究这样的理想模型对于理解自然界中的实际气体是有帮助的:对简单气体模型进行一些必要的修正(例如范德瓦尔斯经验公式),我们可以让它更接近实际气体的行为。其次,我们将对“气体”概念做必要的推广,它的含义并不局限于“无形状有体积的可压缩和膨胀的流体”,用来描述可以用感官感受到的分子气体(箱子里的氦气、水蒸气以及相平衡)。这样的“气体”可以看不见摸不着(光子气体),甚至能够不是实体,仅存在于想象中(声子气体)。

显然,被称为“气体”的系统应当满足“气体”概念的核心性质—— \textcolor{RoyalBlue}{\textbf{\kaishu 稀薄、无相互作用。}}