\chapter{相变的统计力学}\label{cha:相变的统计力学}

在前面的章节中,我们主要处理的是无相互作用粒子组成的系统。对于这类情况,只需知道系统各个组分的能级,就可以直接获悉整个系统的热力学函数。比热、平衡常数、能谱分布等都是隶属于这一类型的主要现象。我们发现这一类这些系统的热力学函数都有一个显著的特征——都是光滑连续的(除了玻色-爱因斯坦凝聚)。

但世界上还有另一类系统等待着统计力学,那就是 \textcolor{RoyalBlue}{\textbf{\kaishu 相互作用系统}}。对于这类情况,当我们研究系统的热力学函数时,会遇到解析上的不连续性和奇异性,并且粒子之间的相互作用再也无法通过坐标变换移除,以至于整个系统的能级与各单一组分的能级之间也就不再有简单的加和关系。在适当条件下,系统大量微观组分之间的相互作用还会变得非常强,从而各组分间呈现出 \textcolor{RoyalBlue}{\textbf{\kaishu 合作}} 行为。当系统的温度达到一个临界 $T_c$ 时,这样的合作行为还会具有宏观效应。

\begin{justification}{\kaishu 反思与质疑}
\kaishu \fontsize{11pt}{16pt}
    \quad\quad 光子气体的玻色-爱因斯坦凝聚的原因并不是相互作用,而是微观粒子的关联性。玻色子的交换对称性使得它们倾向于占据较为靠近的态,而费米子的交换反对称性使得它们倾向于占据较为远离的态(这里比较的对象是既不交换对称也不交换反对称的一般波函数)。这种关联性的结果可以等效为一个吸引势或者排斥势,但并不属于这里讨论的相互作用范畴。

    \quad\quad 另一个问题是:热力学函数是配分函数的偏导数,而配分函数求和的各项很明显是无穷阶可微的,怎么就奇异了呢?如果系统可及的微观态数量有限,那么导数算符的线性性和可微函数的加和性确实可以保证热力学函数处处连续可微,但问题在于有的时候求和的项是可数无限的,此时就会出现函数项级数的求导-求和什么时候可交换的问题,也就是配分函数的一致收敛问题。
\end{justification}

此外,这样的合作现象还具有一些本质的、不依赖于微观细节的特性。例如,系统的响应函数 $\chi $ 在临界点的奇异性,以及系统的各种热力学函数在临界点附近的非解析性。这些特性是相互作用系统的普遍特征,而不是某一特定系统的特征。这类系统的研究是统计力学的一个重要分支,称为 \textcolor{RoyalBlue}{\textbf{\kaishu 临界现象}} 的统计力学。

\textcolor{RoyalBlue}{\textbf{\kaishu 相变}} 集中体现着这些特点。为了简化因相互作用而导致的复杂问题,我们需要引入一些模型,这样的模型需要对相互作用做一定的简化,但仍然保留关于组分间合作行为的基本特性。至于模型能够将相互作用简化至何种程度,就取决于自然界所能呈现的最简单、最容易理解的相互作用是什么。

\textcolor{RoyalBlue}{\textbf{\kaishu 磁学系统}} 是一个比为理想气体添加中心势还要简单的模型——自旋只有一上一下两个状态,相邻自旋方向相同和相反将给出两个不同的相互作用能。如果将这些自旋磁子在一维、二维、三维空间中摆放整齐,并将各自旋之间除了最近邻相互作用能之外的项统统忽略,我们就将得到那个著名、普适、以及优雅的 \textcolor{RoyalBlue}{\textbf{\kaishu Ising 模型}}。


\section{磁学系统的热力学}\label{sec:磁学系统的热力学}

对于以往的理想气体、声子、光子系统,它们的热力学我们是见怪不怪的,也容易理解它们确实具有一定的“热”力学(和环境可以进行热交换,或者我们可以直观地看到、感受到它们的热量)。但对于磁学系统,我们从前从未听说过有什么热力学性质(压强?熵?焓?),只知道有一个 \textcolor{RoyalBlue}{\textbf{\kaishu 居里温度}} 是顺磁性与铁磁性转变的临界温度。

基于我们对磁学系统了解不深的现状,在此有必要建立 \textcolor{RoyalBlue}{\textbf{\kaishu 磁学系统的热力学}},也就是要为(\ref*{sec:微正则系综})一节中提出的各种偏导数赋予独属于磁学系统的物理意义。

为了从宏观的角度描述一个磁学系统,我们发现需要三个参量:温度 $T$ 、磁场 $H$ 、磁矩 $M$ 。可以看到温度和外加磁场均为强度性质,而磁矩则是广度性质。这样一来,对于磁学系统而言,巨正则系综应当作为微观与宏观的桥梁。

首先我们应当从能量守恒定律推导出磁学系统的热力学第一定律,这里不加证明地给出结论:
\begin{equation}\label{equ:磁学系统的热力学第一定律}
    dU = TdS +HdM
\end{equation}
可以看到磁学系统对外做功 $dW = -HdM$ ,代替了理想气体的 $dW = PdV$ 。从这里开始,对第一定律做各种勒让德变换,就可以分别得到磁学系统的各种热力学函数,比如焓、亥姆霍兹自由能、吉布斯自由能等等。这里只给出磁学系统的吉布斯自由能,因为对于我们即将发展的磁学系统的统计力学而言,它是最为重要的热力学函数:
\begin{equation}\label{equ:磁学系统的吉布斯自由能}
    dG = -SdT - MdH
\end{equation}
以及从吉布斯自由能出发得到的系统磁矩表达式:
\begin{equation}\label{equ:磁学系统的磁矩表达式}
    M = -\left(\frac{\partial G}{\partial H}\right)_{T}
\end{equation}

磁学系统的基本热力学函数大概就是以上这些,但一个物理系统我们非常关心这些热力学函数对外界变化的响应,而磁学系统对变化外场 $H$ 的响应就是磁感应系数 $\chi$ :
\begin{equation}\label{equ:磁感应系数}
    \chi = \frac{kT}{N} \left(\frac{\partial M}{\partial H}\right)_{T} = \frac{1}{N} \langle \delta M^2 \rangle
\end{equation}
如 (\ref*{sec:粒子数与能量的涨落}) 节中所讲到的那样,响应函数正比于系统涨落。

\section{Ising 模型}\label{sec:Ising模型}

\subsection{基本模型}

Ising 模型是一个非常简单的模型,它的基本假设是:自旋只有两个状态,即 $S_i = \pm 1$ ,并且自旋之间的相互作用只有最近邻的相互作用能 $J$ ,其余的相互作用能统统忽略。这样一来,Ising 模型一个微观态 $\nu$ 的能量为分为两部分:外场作用和相互作用:
\begin{equation}\label{equ:Ising模型的哈密顿量}
    E_\nu = - \sum_i \mu H S_i-J \sum_{\langle i,j \rangle} S_i S_j
\end{equation}
其中 $\langle i,j \rangle$ 表示求和只对最近邻自旋进行。这里我们假设外场 $H$ 是均匀的,即 $\mu H$ 是一个常数,并且 $J>0$ ,表明自旋倾向于平行排列。

那么很容易写出 Ising 模型的配分函数:
\begin{equation}\label{equ:Ising模型的配分函数}
    Q = \sum_\nu e^{-\beta E_\nu} = \sum_{\{S_i\}} e^{\mu \beta H \sum_i  S_i + \beta J \sum_{\langle i,j \rangle} S_i S_j}
\end{equation}
在后面的讨论中,我们设 $\beta H = K, ~\beta J = K'$ 。从配分函数的表达式中很容易看出系统的磁矩 $M$ 可以写为:
\begin{equation}\label{equ:Ising模型的磁矩表达式}
    \langle M \rangle = \left(\frac{\partial \ln Q}{\partial K}\right)_{K'}
\end{equation}
同时回顾 (\ref*{equ:磁学系统的磁矩表达式}) 式,我们就能建立起热力学函数与配分函数的关系:
\begin{equation}\label{equ:热力学函数与配分函数}
    G = -kT\ln Q(P,H)
\end{equation}


\subsection{熵焓竞争与对称性破缺}

有了以上的讨论,我们在外场 $H  =0$ 的条件下先来说明 \textcolor{RoyalBlue}{\textbf{\kaishu 熵焓竞争}} 怎样影响磁学系统的相变。总体而言,一维 Ising 模型不能相变的原因是熵占优,而二维 Ising 模型能相变的原因是焓占优,并且这样的竞争关系的根源在于边界的维数。我们的思路非常简单:假设系统处于有序状态,然后考虑打乱一部分后体系的自由能变化,如果自由能下降,那么系统就会倾向于无序状态,熵占优,无相变;反之则倾向于有序状态,焓占优,有相变。

\textbf{\kaishu 对于周期边界的一维 Ising 链},一个区域无论大小,永远只有两个端点,所以如果我们将一致向上的 N 个自旋磁子取出一部分磁畴翻转下去,那么相对最初的状态,能量变化为 $4J$ 。由于这样做有 N 个可能的构型,那么熵的变化为 $k\ln N$ 。而 
\[
    \delta G = \delta E - T\delta S = -4J + kT\ln N
\]
显然,我们可以定义一个临界温度 $T_c$ ,在 $2J = kT_c \ln N$ 就是系统相变发生的地方::
\begin{equation}\label{equ:一维Ising模型的临界温度}
    T_c = \frac{4J}{k\ln N}
\end{equation}
可以看到,在热力学极限下,$T_c = 0$ ,也就是说,一维 Ising 模型在任何温度下都不会发生相变,$T = 0$ 是一个平凡解。

\textbf{\kaishu 对于周期边界的二维 Ising 链},一个区域的边界大小是随着区域大小的增加而增加的,所以如果我们将一致向上的 $N$ 个自旋磁子取出一部分磁畴翻转下去,那么相对最初的状态,能量变化 $\delta E \propto N^{1/2}$ 。但由于最多只能有 $N^\alpha$ 种可能的方式达成这一点,那么熵的变化为 $\alpha k\ln N$ 。这个时候,再写出自由能的变化:
\[
    \delta G = \delta E - T\delta S = -N^{1/2} + \alpha kT\ln N
\]
在热力学极限下,第一项就会超过第二项,也就是能量占优,磁畴的翻转变得困难,所以更倾向于保持同向,也就是相变。

这也就解释了为什么我们实际观察到体系的对称性要小于系统哈密顿量的对称性,也就是 \textcolor{RoyalBlue}{\textbf{\kaishu 对称破缺}} 的问题。磁学系统哈密顿量虽然具有自旋翻转的对称性(系统所有自旋全部翻转,总能量不变),但要想在实际中体现这一点,必须要求这个态要能够自由地变换到另一个对称态。但如前所见,这样的翻转动作将会面临巨大的能垒,且在热力学极限下能垒高度为正无穷,所以动力学上不能发生。用一个公式来表示就是:
\begin{equation}\label{equ:对称性破缺}
    \langle M \rangle  = \lim_{H\rightarrow 0}  \lim_{N\rightarrow \infty} \left(-\frac{\partial G}{\partial H}\right)_T \neq 0  
\end{equation}
在 $N$ 有限的情况下,系统可以随意翻越能垒,所以磁矩可以为 $0$ ;现在为系统增加一个外场,那么系统将会聚集在某一边。现在让 $N\rightarrow \infty$ ,把门关上,系统就被困在了一个状态,纵使外场消失,系统即使想回去,也回不去了。所以宏观磁矩不为 $0$ ,这就是对称性破缺的本质。