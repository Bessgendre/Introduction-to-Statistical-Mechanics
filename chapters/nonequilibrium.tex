\chapter{非平衡统计力学}

平衡态与非平衡态的区别,引用前面的内容,由此引出非平衡分布及其演化(Fokker-Planck方程)

\section{再论刘维尔定理}

定义刘维尔算符,反自伴性

力学量演化 vs 分布演化 (薛定谔绘景与海森堡绘景)

非平衡平均与平衡平均都是系综平均,只是分布有差别(薛定谔)或者力学量有差别(海森堡)

系综是概率空间,力学量是函数,分布是概率测度

什么意义上的概率分布,谁是随机变量,谁是事件,谁是随机过程

\section{分布的演化}
薛定谔绘景
\subsection{布朗运动的爱因斯坦理论}
独立和,概率密度的推导以及使用斯特林公式后的渐进形式

引入扩散方程,指出扩散方程的解和这里相同,从而指出概率的微观解释与宏观解释的等价性将导致随机过程与扩散方程的等价性

\subsection{Fokker-Planck方程}


\section{力学量的时间关联}
海森堡绘景

\subsection{协方差与关联}

\subsection{稳态性质}


\section{布朗运动的朗之万理论}

\section{涨落的谱分析:维纳-欣钦定理}

\section{线性响应理论}

\subsection{响应函数}

\subsection{昂萨格回归假说}

\section{涨落耗散定理}
